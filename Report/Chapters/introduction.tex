\chapter{Introduction}
The pronunciation is one of the hardest part of learning a language among all the other components, such as grammar rules and vocabulary. To achieve a good level of pronunciation, non native speakers have to study and constantly practice the target language for a incredible amount of hours. In most cases, when students are learning a new language, the teacher is not a native speaker in which implies that the pronunciation may be influenced by the country where he or she comes from, since it is a normal consequence of second learning language \cite{derwing2005second}. In fact, \cite{medgyes2001teacher} states that the advantages of having a native speaker as a teacher, lies in the superior linguistic competences, especially the usage of the language more spontaneously in different communication situations. Pronunciation falls into those competences underlying a base problem in teaching pronunciation at school.  \\

\noindent The basic questions we tried to answer in this work are:
\begin{compactitem}
    \item[1)] Why is pronunciation so important?
    \item[2)] What are the most effective methods for improving the pronunciation?
    \item[3)] What is the research state-of-art and what can we do to make it better?
\end{compactitem}

\vspace*{1em}

\noindent The first question is fairly easy to answer. There are two reasons to claim why pronunciation is important: \textit{(i)} it helps to acquire the target language faster and \textit{(ii)} being understood.
Regarding the first point, earlier a learner masters the basics of pronunciation, the faster it will become fluent. The reason is because \textit{critical listening} with a particular focus on hearing the sounds will lead to gain fluency in speaking the language.
The second point is \textbf{crucial} when working with other people, especially at these days where both in school and business the environment is often multicultural. Pronunciation mistakes may lead the person to being misunderstood affecting the results of a project for example. \\

\noindent With these statements in mind, \cite{gilakjani2011pronunciation} gives suggestions on how a learner can effectively improve the pronunciation. Four important ways are depicted: \textit{Conversation} is the most relevant approach to improve pronunciation, although, a supervision of an \textit{expert guidance} that corrects the mistakes is fundamental during the process of learning. At the same time, learners have to be pro-active to have conversation with other native speakers in such a way to constantly practicing. \textit{Repetitions} of pronunciation exercises is another important factor that will help the learner to be better in speaking. As last, \textit{Critical listening}, that we also mentioned above, amplify the opportunity of learning the way native speakers pronounce words. In particular, for a learner is important to understand the difference when he or she is pronouncing a certain sentence with the one said by the native speaker. This method is very effective and is important for understanding the different sounds of the language and how a native speaker is able to reproduce them \cite{rost2014listening}.

\noindent An important factor while learning a second language is to have a feedback about improvements. Teachers are usually responsible to judge the way the learners' progress. In fact, when teaching pronunciation, often it is used to draw the intonation and the stress of the words in such a way that the learner is able to see how the utterances should be pronounced. The \textit{British Council} shows this practice \cite{bbc_stress}. The usage of visual feedbacks is the key of learning pronunciation and it is the main feature of this research. \\

\noindent In the computer science field, some works have been previously done regarding pronunciation. For instance, \cite{edge2012tip} helps learners to acquire the tonal sound system of Mandarin Chinese through a mobile game. Another example is \cite{head2014tonewars} in which the application provides a platform where learners of Chinese language can interact with native speakers and challenging them to a competition of pronunciations of chinese tones. \\

\noindent The idea behind this project is based on the fact that people need to keep practicing their pronunciation to have a significant improvement as well as they need immediate feedbacks to understand if they are going in the right direction or not. The approach we used is based on these two factors and we designed the system to be as useful and portable as possible. The mobile application is where the user will test the pronunciation, a server through a machine learning technique will compute the similarity between the user's pronunciation and the native speaker's one and the results will be displayed on the phone. \\

\noindent We started collecting data from \textit{American Native Speakers} in which we asked to pronounced a set of most used idioms and slangs. Each candidate had to repeat the same sentence four times trying to be as much consistent as possible. After we gathered the data, a preprocessing step is due since we are seeking for specific features such as voice-stress, accent, intonation and formants. This part is actually divided in two parts as well, where we used \textbf{PRAAT}\cite{boersma2010p} to extract the features from each audio file and then we used \textbf{MATLAB}\footnote{http://www.mathworks.com} to average the four audio signals that each candidate said. After the averaging step, there is a smoothing process in which we applied the \textit{Exponential Filter} to each signal. A \textit{resampling} was necessary since not all the signals have the same length in time. The next step consisted in the so called \textbf{force alignment} where for each audio file we extract the \textit{phonemes} that have been pronounced. \\
\noindent The same treatment is given to the audio signals given by the final user in order to compare the pronunciation. \\

\noindent The machine learning part is composed by a \textit{deep neural network} trained using a statistical modeling method called \textbf{conditional random fields}. The system is trained using the features extracted during the data processing phase, coupled with the phonemes retrieved using the force alignment. The phonemes are used as labels for the classification in which we indeed predict. To estimate the overall error between the native pronunciation and the user, we use a method called \textbf{Word Error Rate} (WER), a common method metric for measuring the performance of a speech recognition system. In addition to \textit{WER}, we used \textbf{Dynamic Time Warping} (DTW) to evaluate the similarity between features. These results will allow the user to have a better understanding of the differences for each characteristic. \\
\noindent DTW is not a part of the neural network, but it's a standalone algorithm. \\

\noindent When the server has computed the similarity, a score with the evaluation of voice-stress, accent, intonation and formants is given to the user. Those features are not the only one available. In fact, the application is capable to show the graph for each feature of the user's pronunciation and the one of the native speaker. In this way the user has a clear understanding of how he or she should \textbf{adjust} the way the utterance should be pronounced accordingly. \\

\noindent To measure the accuracy of our model, we asked 10 native speakers to rate the pronunciation of 10 non native speakers on 5 different sentences. The range of evaluation was between 0 and 10. From here we averaged all the rates and compared with the results obtained by the automatic system.
