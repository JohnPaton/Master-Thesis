%!TEX root = ../thesis.tex

\chapter{Conclusions}
\label{chap:Conclusions}

In conclusion, we can claim that the application has a lot of potential and there is a lot of room for improvement. Looking at the results,  users did not significantly improve the pronunciation with the tools we provided. After a careful analysis of the data, we think that users did not use the application enough for seeing an actual improvement. There could be multiple reasons for that: few sentences available, few indications on improving the pronunciation, long waiting time in order to get feedback, etc. \\
\noindent We also think that users did not understand why we included features such as the \textit{critical listening} and the \textit{history page}. One reason could be that, despite linguistic research claiming that these two features are very useful during the learning process, for pronunciation purpose, users need something else. Discovering these actual needs, goes beyond the scope of our research. To be sure, we understood that these two methods are not particularly important for this process. \\
\noindent The results related to the feedback page were not particularly positive. We think that a more careful study on how to deliver the feedback to user is necessary in the future, or rather, finding a way to give clearer directions on how to improve the pronunciation. However, the aim of the project was to avoid the usage of words in order to give feedback, but simply to use visual information. We find out that this type of information is very hard to deliver and in the future, a better/different approach is definitely necessary. \\

\noindent Despite some low scores, our testers appreciated the prototype and the way we delivered the idea. People need this kind of application to learn and improve languages, and the indicated interest regarding the usage of mobile devices to become well-versed in other languages is very high.
