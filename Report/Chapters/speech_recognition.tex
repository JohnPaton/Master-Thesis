\chapter{Speech Recognition}
\label{chap:Speech Recognition}
Speech recognition is a sub-field of machine learning in which allows a computer program to extract and recognize words or sentences from a human being language, and converting them back to a machine language. Advance techniques nowadays, permits to understand natural speech for executing tasks. Google Voice Search\footnote{\url{https://www.google.com/search/about/}} and Siri\footnote{\url{http://www.apple.com/ios/siri/}} are two examples of very advance speech recognition softwares with the capability of understanding natural language.

\section{The Problem of Speech Recognition}
\label{sec:The Problem of Speech Recognition}
Human languages are very complex and different among each other. Despite they might have a well-structured grammar, automatically recognition is still a very difficult problem since people have many ways to say the same thing. In fact, spoken language is different from written one because the articulation of verbal utterance is less strict and complicated. \\
The environment in which the sound is taken has a big influence on the speech recognition software because it introduces a \textit{unwanted} amount of information in the signal. For this reason it is important that the system is capable of \textit{identifying} and \textit{filtering out} this surplus of information \cite{forsberg2003speech}. \\

\noindent Another interesting set of problems are related to the speaker itself. Each person has a different body that means there are a variety of components that the recognition system has to take care of in such a way to be able to understand correctly. Gender, vocal tracts, speaking style, speed of the speech, regional provenience are fundamental parts that have to be taken in consideration when building the \textit{acoustic model} for the system. Despite these features are unique for each person, there some common aspects that will be used to construct the model. The acoustic model represents the relationship between the acoustic signal of the speech and the phonemes related to it. \\

\noindent Ambiguity represents the major concern since natural languages have inherited it. In fact, it may happen that in a sentence we are not able to discriminate which words are actually intended \cite{forsberg2003speech}. In speech recognition there are two types of ambiguity: \textit{homophones} and \textit{word boundary ambiguity}. \\
Homophones refers to those words that are spelled in a different way but they \textbf{sound} the same. Generally speaking, these words are not correlated to each other but it happened that the sound is equivalent. Word boundary ambiguity instead, it \textit{occurs when there are multiple ways of grouping phones into words}\cite{forsberg2003speech}.

% CMU Sphinx4
\section{Architecture}
\label{sec:speech_rec_Architecture}
Generally speaking, a speech recognition system is divided in three main components: the \textbf{Feature Extraction} (or Front End), the \textbf{Decoder} and the \textbf{Knowledge Base} (KB). In \ref{fig:speech_architecture} the KB part is represented by the three sub-blocks called \textit{Acoustic Model}, \textit{Pronunciation Dictionary} and \textit{Language Model}. The \textit{Front End} takes as in input the voice signal where it is analyzed and converted in the so called \textit{Features Vectors}. This last is the set of common properties that we discussed in \ref{ch:english_language}. From here we can say that $\textbf{Y} 1:N = y_{1},..., y_{N}$ where $Y$ is the set of features vectors. \\
The second step consists in feeding the \textit{Decoder} with vectors we obtained from the previous step, attempting to find the sequence of words $\textbf{w} 1:L = w_{1}, ... , w_{L}$ that have most likely generated the set $Y$\cite{gales2008application}. The decoder tries to find the likelihood estimation as follows:

\begin{equation}
	\widehat{w} = \underset{w}{arg max} P(\textbf{w}| \textbf{Y})
\end{equation} 

\noindent The $P (w|Y)$ is difficult to find directly\footnote{There is discriminate way of finding the estimation directly as described in \cite{gales2007discriminative}}, but using Bayes' Rules we can transform the equation above in

\begin{equation}
	\widehat{w} = \underset{w}{arg \; max} P (\textbf{Y}|\textbf{w}) P(\textbf{w})
\end{equation}

\noindent in which the probability $P(Y|w)$ and $P(w)$ are estimated by the \textit{Knowledge Base} block. In particular, the \textit{Acoustic Model} is responsible to estimate the first one whereas the \textit{Language Model} estimates the second one. \\
\noindent Each word \textbf{w} is decomposed in smaller components called \textit{phones}, representing the collection of phonemes $\textbf{K}_{w}$ (see \ref{ch:english_language}). We can describe the \textit{pronunciation} as $\overset{(w)}{\textbf{q}_{1:K_{w}}} = q_{1}, ...., q_{K_{w}}$. The likelihood estimation of the sequence of phonemes is calculated by a \textbf{Hidden Markov Model} (HMM). In the section, a general overview of HMM is given. We are not going to discuss a particular model because every speech recognition system uses a variation of the general HMM chain. \\

\begin{figure}[!ht]
	\centering
	\includegraphics[scale=0.8]{Figures/speech_Architecture.png}
	\caption{HMM-Based speech recognition system \cite{gales2008application}}
	\label{fig:speech_architecture}
\end{figure}

\section{Hidden Markov Model}
\label{sec:hmm}
\noindent A definition given by \cite{eddy1996hidden} is the following: \textit{"An Hidden Markov Model is a finite model that describes the probability distribution over an infinite number of possible sequences"}. Each sequence is determined by a set of \textit{transition probabilities} in which describes the transitions among states. The \textbf{observation} (or outcome) of each state is generated based on the associated probability distribution. From an \textit{outside} perspective, the \textit{observer} is only able to see the outcome and not the state itself. Hence, the states are considered \textbf{hidden} which leads to the name Hidden Markov Model \cite{def_hmm}. \\

\noindent An HMM is composed by the following elements:

\begin{itemize}
	\item The number of states (N)
	\item The number of observations (M), that becomes infinite if the set of observations is contiguous
	\item The set of transition probabilities, $\Lambda = \{ a_{ij}\}$
\end{itemize}

The set of probabilities is defined as follow:
\begin{equation}
\label{eq:transition_probabilities}
a_{ij} = p \, \{ \, q_{t+1} = j \, | \, q_{t} = i \, \}, \, \, \, 1 \leq i,j \leq N, 
\end{equation}

\noindent where $q_{t}$ is the state we are currently in and $a_{ij}$ represent the transition from state $i$ to $j$. 
Each transition should satisfy the following rules:

\begin{subequations}
	\label{eq:stochastic_rules}
	\begin{align}
	a_{ij} \geq 1, \, \, \, 1 \leq i,j \leq N, \\
	\sum_{j=1}^{N} a_{ij} = 1, \, \, \, 1 \leq j \leq N
	\end{align}
\end{subequations}

\noindent For each state $S$ we can define the probability distribution $S = \{s_{j}(k)\}$ as follow:

\begin{equation}
s_{j}(k) = p \, \{\, o_{t} = v_{k} \, | \, q_{t} = j \, \}, \, \, \, 1 \leq j \leq N, \,\, 1 \leq k \leq M
\end{equation}

\noindent where $v_k$ is the $k^{th}$ observation whereas $o_{t}$ is the outcome. Furthermore, $b_{j}(k)$ must satisfy the same stochastic rules described in \ref{eq:stochastic_rules}. \\

\noindent A different approach is made when the number of observations is infinite. In fact, we are not going to use a set of discrete probabilities but instead a continuous probability density function. Given that, we can define the parameters of the density function by approximating it by a weighted sum of $M$ Gaussian distributions $\varphi$ \cite{def_hmm}. We can describe the function as follow:

\begin{equation}
	s_{j}(o) = \sum_{m = 1}^{M} c_{jm}\varphi (\mu_{jm}, \Sigma_{jm}, o_{t})
\end{equation}

\noindent where $c_{jm}$ is the weighted coefficients, $\mu_{jm}$ is the mean vector and $\Sigma_{jm}$ is the covariance matrix. The coefficients should satisfy the stochastic rules in \ref{eq:stochastic_rules}. \\
\noindent We can then define the initial state distribution as $\pi = \{\pi_{i}\}$ where 

\begin{equation}
	\pi_{i} = p \, \{ q_{I} = i \, \}, \,\,\,\, 1 \leq i \leq N
\end{equation} 

\noindent Hence, to describe the HMM with the discrete probability function we can use the following compact form
\begin{equation}
\label{eq:hmm_discrete}
	\lambda = (\Lambda, S, \pi )
\end{equation}

\noindent whereas to denote the model with a continuous density function, we use the one described in \ref{eq:hmm_density}
\begin{equation}
\label{eq:hmm_density}
\lambda = (\Lambda, c_{jm}, \mu_{jm}, \Sigma_{jm}, \pi )
\end{equation} 

\subsection{Assumptions}
\label{sub:assumptions_hmm}
The theory behind HMM requires three important assumptions: the \textbf{Markov assumption}, the \textbf{stationarity assumption} and the \textbf{output independence assumption}.

\subsubsection{The Markov Assumption}
The Markov assumption assumes that the following state depends only from the state we are currently in as given in \ref{eq:transition_probabilities}. The result model is also referred as \textit{first order} HMM. Generally speaking though, the decision of the next coming state might depend on \textbf{n} previous states, leading to a $n^{th}$ HMM order model. In this case, the transition probabilities is defined as follow:

\begin{equation}
\label{eq:transition_nth_order}
a_{i_{1}i_{2}...i_{n}j} = p\, \{\, q_{t+1} = j \,|\, q_{t} = i_{1}, \, q_{t-1} = i_{2}, ... , \, q_{t-k+1} = i_{k} \, \}, \,\,\,\, 1 \leq i_{1},i_{2}, ... ,i_{k}, j \leq N
\end{equation}


\subsubsection{The Stationary Assumption}
The second assumption states that the transition probabilities are \textit{time-independent} when the transitions occur. This is defined by the following equation for any $t_{1}$ and $t_{2}$:

\begin{equation}
	p \, \{\, q_{t+1} = j \,|\, q_{t_{1}} = i\, \}\, = \,p \,\{\, q_{t_{2}+1} = j \,|\, q_{t_{2}} = i\, \}
\end{equation}

\subsubsection{The Output Assumption}
The last assumption says that the current observation is statistically independent from the previous observations. Let's consider the following observations:

\begin{equation}
	O = o_{1}, o_{2}, ... , o_{T}
\end{equation}

Now, recalling \ref{eq:hmm_discrete}, it is possible to formulate the assumption as follow:

\begin{equation}
	p \, \{\, O \, |\, q_{1},q_{2}, ... , q_{T}, \lambda \,\}\, = \, \prod_{t = 1}^{T} p \, \{ \, o_{t} \, | \, q_{t}, \, \lambda \,\}
\end{equation}

\section{Evaluation problem}


\subsection{Forward Algorithm}


\subsection{Backward Algorithm}


\section{Viterbi Algorithm}
\label{sec:viterbi}
Decoding part

\section{Maximum likelihood estimation}
\label{sec:mle}
Training the HMM



% GMM classifier
\section{\Naive Bayes and Gaussian models for classification}
\label{sec:Gaussian Classifiers and Distance Measures}
