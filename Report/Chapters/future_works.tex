\chapter{Future Works}
\label{ch:Future Works}

Given the results many other applications can be extracted from this prototype. Smartwatches for example, are becoming the next hot-platform for developing new applications. In fact, it is possible to extend this product in such a way that a user can practice day-by-day by simply using the internal microphone of the smartwatch. The procedure and the time taken for the whole process is less then using a common smartphone. Of course, the whole feedback system has to be redesigned and scaled to be able to fit the information in a smaller screen. \\

\noindent Another interesting way for pushing the limits of this application, is to make it more challenging more like a video game. In fact, provide the opportunity for the user to challenge other users should give a psychological boost for improving the pronunciation and be better than other competitors. Thus, the usage of achievements, objectives, etc. will involve the user in a completely different experience but still with the intent of improving the pronunciation. \\

\noindent \textit{Google Glass}\footnote{https://www.google.com/glass/start/}, \textit{Microsoft HoloLens}\footnote{https://www.microsoft.com/microsoft-hololens/en-us}, \textit{Oculus Rift}\footnote{https://www.oculus.com/en-us/} and other augmented reality devices, could be used for language learning process. The user will then be involved in an experience that would be closer to an actual lecture with a qualified teacher. Using a virtual assistant and a complex AI system, it would be possible to reproduce this old, but still very effective, way of learning. At the same time, interaction with other users that have the same application and device, would be incredibly effective to train not only the pronunciation but also grammar, reading-comprehension and conversation. \\

\noindent The number of possible and future applications is incredibly large. These were simple example of how we can use the new coming technology in the world of learning languages.
