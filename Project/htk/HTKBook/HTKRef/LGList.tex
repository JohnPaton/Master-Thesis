%
% HLMBook - Steve Young    07/01/97
%
% Updated - Gareth Moore   15/01/02
%

\newpage
\mysect{LGList}{LGList}

\mysubsect{Function}{LGList-Function}

\index{lglist@\htool{LGList}|(}
This program will list the contents of one or more \HLM\ gram files.
In addition to printing the whole file, an option is provided to print
just those $n$-grams containing certain specified words and/or ids.  It
is mainly used for debugging.

\mysubsect{Use}{LGList-Use}

\htool{LGList} is invoked by typing the command line
\begin{verbatim}
   LGList [options] wmapfile gramfile ....
\end{verbatim}
The specified gram files are printed to the output. The $n$-grams are
printed one per line following a summary of the header information.
Each $n$-gram is printed in the form of a list of words followed by the
count.

Normally all $n$-grams are printed. However, if either of the options
\texttt{-i} or \texttt{-f} are used to add words to a \textit{filter
list}, then only those $n$-grams which include a word in the filter list
are printed.

The allowable options to \htool{LGList} are as follows

\begin{optlist}
 \ttitem{-f w} Add word \texttt{w} to the filter list.  This option
     can be repeated, it can also be mixed with uses of the
     \texttt{-i} option.
 \ttitem{-i n} Add word with id \texttt{n} to the filter list.  This
     option can be repeated, it can also be mixed with uses of the
     \texttt{-f} option.
\end{optlist}
\stdopts{LGList}

\mysubsect{Tracing}{LGList-Tracing}

\htool{LGList} supports the following trace options where each
trace flag is given using an octal base
\begin{optlist}

\ttitem{00001}  basic progress reporting. 
\end{optlist}
Trace flags are set using the \texttt{-T} option or the  \texttt{TRACE} 
configuration variable.
\index{lglist@\htool{LGList}|)}







